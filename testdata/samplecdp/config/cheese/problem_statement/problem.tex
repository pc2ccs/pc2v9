\problemname{Cutting Cheese}

\illustration{0.4}{cheese}{Picture by Jon Sullivan via Wikimedia Commons}%
Of course you have all heard of the International Cheese Processing Company. Their machine for cutting a piece of cheese into slices of exactly the same thickness is a classic. Recently they produced a machine able to cut a spherical cheese (such as Edam) into slices -- no, not all of the same thickness, but all of the same weight! But new challenges lie ahead: cutting Swiss cheese.

Swiss cheese such as Emmentaler has holes in it, and the holes may have different sizes. A slice with holes contains less cheese and has a lower weight than a slice without holes. So here is the challenge: cut a cheese with holes in it into slices of equal weight.

By smart sonar techniques (the same techniques used to scan unborn babies and oil fields), it is possible to locate the holes in the cheese up to micrometer precision. For the present problem you may assume that the holes are perfect spheres.

%The density of the cheese to be cut is $1.1$ g/cm$^3$.
%Assume that the cheese is uniformly dense.
Each
uncut block has size $100 \times 100 \times 100$ where each dimension is measured in millimeters.  Your task is to cut
it into $s$ slices of equal weight. The slices will be $100$~mm
wide and $100$~mm high, and your job is to determine the thickness of each
slice. 


\section*{Input}

The first line of the input contains two integers $n$ and
$s$, where $0 \leq n \leq 10\,000$ is the number of holes in the
cheese, and $1 \le s \le 100$ is the number of slices to cut. The next
$n$ lines each contain four positive integers $r$, $x$, $y$, and $z$ that describe a
hole, where $r$ is the radius and $x$, $y$, and $z$ are
the coordinates of the center, all in micrometers.  

The cheese block occupies the points $(x,y,z)$ where $0 \le x,y,z \le
100\,000$, except for the points that are part of some hole.  The cuts are made perpendicular to the $z$ axis.

You may assume that holes do not overlap but may touch, and that the holes are fully contained in the cheese but
may touch its boundary.


\section*{Output}

Display the $s$ slice thicknesses in millimeters,
starting from the end of the cheese with $z=0$.
Your output should have an absolute or relative
error of at most $10^{-6}$.

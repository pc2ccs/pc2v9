
\problemname{Evolution in Parallel}

It is 2178, and alien life has been discovered on a distant planet. There seems to be only one
species on the planet and they do not reproduce as animals on Earth do.  Even more amazing, the genetic makeup of every single organism is identical!

The genetic makeup of each organism is a single sequence of nucleotides. The
nucleotides come in three types, denoted by `A' (Adenine), `C' (Cytosine), and `M' (Muamine).
According to one hypothesis,
evolution on this planet occurs when a new nucleotide is inserted somewhere into the genetic
sequence of an existing organism.  If this change is evolutionarily advantageous, then organisms
with the new sequence quickly replace ones with the old sequence.

It was originally thought that the current species evolved this way from a single, very simple
organism with a single-nucleotide genetic sequence, by way of mutations as described above.
However, fossil evidence suggests that this might not have been the case. Right now, the research
team you are working with is trying to validate the concept of ``parallel evolution'' -- that there might actually have been two
evolutionary paths
% separate species
evolving in the fashion described above, and eventually both paths evolved to the single species present on the planet today.  Your task is to 
 verify whether the parallel evolution hypothesis is consistent with the genetic material found in the fossil samples gathered by your team.


\section*{Input}

%In the first line of the input the number of test cases $1\le T\le 40$ is given.
The input begins with a number $n$ ($1\le n\le 4\,000$) denoting the number of nucleotide sequences found
in the fossils. The second line describes the nucleotide sequence of the species currently living on the
planet. Each of the next $n$ lines describes one nucleotide sequence found in the fossils.

Each nucleotide sequence consists of a string of at least one but no more than
$4\,000$ letters. The strings contain only upper-case letters {\tt A}, {\tt C}, and {\tt M}.
All the nucleotide sequences, including that of the currently live species, are distinct.


\section*{Output}

%You must output a solution to each of the test cases.
%If it is impossible for all the fossil material to come from two species
%that both eventually evolved to the currently live species,
%display the word {\tt impossible}.
%If it is possible for the hypothesis to be correct,
Display an example of how the nucleotide sequences in the fossil record participate in two evolutionary paths.
The example should begin with one line containing
two integers $s_1$ and $s_2$, the number of nucleotide sequences in the fossil record that participate in the first path and second path, respectively. This should be followed by $s_1$ lines containing the sequences attributed to the first path, in chronological order (from the earliest),
and then $s_2$ lines containing the sequences attributed to the second path,
also in chronological order.
If there are multiple examples, display any one of them.
If it is possible that a sequence could appear in the genetic history of both species, your example should assign it
to exactly one of the evolutionary paths.

%You must output a solution to each of the test cases.
If it is impossible for all the fossil material to come from two evolutionary paths,
display the word {\tt impossible}.


